\chapter{Limitations \& future prospects}

\label{chapter6}

\section{Conclusion}

The actual spin current reading is significantly low, upto the point that it gets hidden under background noise (in the form of stray potentials).
Hence from an experimental point of view, we do not observe any meaningful correlation between the charge current in Pt and observed voltage measurement (see \cref{chapter5}).

\section{Limitations}

\subsection{Generation of spin imbalance in Pt}

We refer back to \cref{fig:layer-workflow}, through which we evaluate the setup for reasons of low measurement readings (noisy) and proceed to suggest a better setup for measuring with greater efficacy in our next experimental run.

As seen in the figure, when \( J_c \) is passed through the first arm of the \Hst, a major portion of the current flows through Cu, it being an excellent conductor.
This is however, not desirable since the generation of spin imbalance occurs due to Pt (because of its high SOC) and with the current setup, the amount of spin imbalance is reduced.

In an ideal setup, we would require most of the current to move through Pt, generating a "good" quantity of spin imbalance.
The corresponding spin polarized electrons would then travel through the Cu layer, via the link and convert back to charge current.


\subsection{Poor conversion efficiency}

We mentioned earlier in \cref{subsec:efficacy}, that as opposed to typical experimental studies involving spin current detection which only involves a one-way conversion to charge current, we proceed to supply a charge current, followed by conversion to spin current, which is again followed by conversion to charge current, leading to a two-way conversion pathway.

Each time such a conversion occurs, the efficiency is poor (even if Pt has good SOC), leading to a poor measurement data for detecting spin current. This low measurement is further diminished in our setup, since we go through a two-way conversion pathway.

\subsection{Complications with multilayered systems}

\label{subsec:multilayer}

Multilayered samples have the issue of current moving disproportionately amongst layers as opposed to the desired quantity for a good measurement.
These complications arise due to movement of current through the interface of two materials (this might cause some dissipation) or due to shunting (which happens in our sample, due to both the materials being metals), which is undesirable and causes poor efficacy in measurement of spin current \cite{maekawa2017spin}.

Ideally, one would like to use one material, which has good SOC to facilitate the conversion to spin current and good spin diffusion length \( \delta_s \), onto which both generation and detection can be done.

\subsection{Thickness of layer}

\label{subsec:thickness}

Using a thicker layer in our sample introduces a reduction in the current density along with an increase in resistance against current.
We strongly try to avoid the former, since a greater current density yields a greater spin imbalance via SHE and consequently, voltage reading that can be accurately measured.
This can be seen via \cref{eq:thin-slab}, where a thinner slab yields better spin Hall voltage.

\section{Scope for improvement}

As mentioned in sections \ref{subsec:multilayer} and \ref{subsec:thickness}, during the next run of the experiment, we aim to use a bilayer instead of a trilayer, hence reducing the number of NM-HM metal interfaces by one.

Also, the only purpose of the arm-link in the \textsc{H}-structure is to enable the flow of spin-polarized electrons in the form of spin current upto a greater distance, facilitated by the high spin diffusion length of Cu.
Therefore, the support outer layers of Pt can be removed to concentrate all the spin current to flow through the middle layer of Cu.

\section{Future prospects}

Spintronics has a wide potential for drastically changing the technological industry for the better.
As seen in \cref{chapter1}, Moore's law poses an alarming threat to conventional electronics and the technological sphere as we know it \cite{moore1998cramming}.
This upcoming field based on the spin of the electronic can be seen to have a vast potential is improving this dire situation and help produce devices much better in terms of power consumption and switching times \cite{wolf2001spintronics}.

Our study gives us a glimpse at the practical difficulties involved in making performant spin-based devices.
For example, making transistors that work on spin is challenging as generation and detection of spin current is not very trivial \cite{vzutic2004spintronics}.
