\chapter{Limitations \& future prospects}

\label{chapter6}

\section{Inference}

As seen from \cref{chapter5}, no inferable data could be obtained due to the voltage reading being below the threshold of our measuring apparatus i.e. a nano-voltmeter.

\section{Limitations}

% Mention the seepage of current into Cu, which we don't want and the advantage of having a single layer instead of bilayer sample.

\subsection{Generation of spin imbalance in Pt}

We refer back to \cref{fig:layer-workflow}, through which we evaluate the setup for reasons of low measurement readings (noisy) and proceed to suggest a better setup for measuring with greater efficacy in our next experimental run.

As seen in the figure, when \( J_c \) is passed through the first arm of the \Hst, a major portion of the current flows through Cu, it being an excellent conductor.
This is however, not desirable since the generation of spin imbalance occurs due to Pt (because of its high SOC) and with the current setup, the amount of spin imbalance is reduced.


\subsection{Poor conversion efficiency}

We mentioned earlier in \cref{subsec:efficacy}, that as opposed to typical experimental studies involving spin current detection which only involves a one-way conversion to charge current, we proceed to supply a charge current, followed by conversion to spin current, which is again followed by conversion to charge current, leading to a two-way conversion pathway.

Each time such a conversion occurs, the efficiency is poor (even if Pt has a good SOC), leading to a poor measurement data for detecting spin current. This low measurement is further diminished in our setup, since we go through a two-way conversion pathway.

\subsection{Single layer v.s. Multi layer}


\subsection{Thickness of layers}
