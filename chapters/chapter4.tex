\chapter{Methodology}

\label{chapter4}

In this chapter, we discuss about the techniques and methods used to devise an experiment involving a Pt-Cu bilayer system and subsequently investigate the effect of spin current via SHE.

\section{Experimental plan}

In earlier studies, it has been shown that a pure spin current can be generated and manipulated via a heavy metal through SHE \cite{hirsch1999spin,sinova2004universal,zhang2000spin}. The detection of this spin current can only be done via conversion into charge current (which is measurable) using ISHE. This generally involves a ferromagnetic layer or a magneto-optical method \cite{kimura2007room,li2019spin,stamm2017magneto}.

In our study, we intend to explore the detection of spin current in a NM/HM bilayer system without a magnetic layer.

\section{Preparation of sample}

\begin{figure}
    \centering
\begin{tikzpicture}
    \draw (0,0) -- (5,0) -- (5,3) -- (0,3) -- cycle;
    \draw (0,1) -- (5,1);
    \draw (0,2) -- (5,2);
\end{tikzpicture}
    \caption{Schematic diagram of trilayer sample}
\end{figure}
