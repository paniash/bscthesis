\chapter{Introduction}

\label{chapter1}

\section{Spintronics: an emerging field}

Spintronics is an emerging field in the technological sphere, which aims to complement the traditional role of electrons as charge carriers and exploit its properties related to spin.
Conventionally, all electronic devices use binary digits 0 and 1 for store information in the form of the electron's charge.
All such devices operate on components called transistors which are responsible for the flow of electric current to facilitate the processing of information.

Moore's law predicts that the number of transistors on such devices doubles every eighteen months \cite{moore1998cramming}.
However, this consequently also means that the size of individual transistors decreases.
This raises a problem as the size of the transistors decrease, quantum phenomena become more dominant at such length scales.
This difficulty is compensated by the rise of spintronics, which is seem as an alternative technology, based on the spin of the electron along with its charge to store and carry information.

The evolving field has its future prospects in making energy efficient devices with high storage density and low power consumption \cite{Yakout_2020}.

\subsection{Why is spin current so important?}

Pure spin current refers to the flow of a net angular momentum where there is no measurable charge current (the type of current that we can measure using ammeters) and plays a crucial role in the field of spintronics.

Conventional devices are based on the charge of the electron and do not depend on the its spin.
However, newer devices are being built that exploit this very property \cite{wolf2001spintronics,vzutic2004spintronics}.
Such devices would require low power to operate and have much faster switching times (switching 0s to 1s and vice versa) than conventional devices, simply because manipulating spin is much faster and costs less power.

\section{Aim and motivation behind the experiment}

Past studies have demonstrated that it is possible to detect spin current via a one-way conversion pathway involving spin Hall effect and to be able to use magneto-optical detection techniques to measure spin current \cite{Stamm_2017, valenzuela2007electrical}.

In contrast to the above however, in our experiment we intend to use conventional electrical detection techniques to measure spin current (refer \cref{subsec:spin-current}) directly using charge current as our input to the bilayer sample.
This involves a two-way conversion pathway as opposed to the former method, but provides the opportunity for more simpler means of measurement.

In our study, we present the intricacies and results yielded from the experiment along with the difficulties faced and scope for improvement for the experiment.
